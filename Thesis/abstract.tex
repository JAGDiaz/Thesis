% You insert your abstract in the space below.


Machine Learning is a useful tool for many problems in modern applied mathematics, especially given the
availability of data and recent advancements in data-driven methods of model discovery and prediction such 
as the family of Koopman derived
methods known as Dynamic Mode Decomposition. As with any such methods, quantifying the 
effectiveness of training is a meaningful question to ask when standard metrics do not give a concrete 
answer. The evolution of a Neural Network under training can be thought of as an extremely high-dimensional
dynamical system in and of itself, given that the number of trainable parameters regularly reaches into the 
hundreds, thousand, and (in our case) the hundreds of thousands. Classical methods of characterizing dynamical
systems were not designed to handle such high-dimensional data, so we propose that data scientific methods must 
be devised to describe Machine Learning model evolution with data. In this thesis, we explore such methods 
from a statistical, probabilistic,
and information theoretic point of view to find classifiers of good training among many different trainings
of the Deep Learning Enhanced Dynamic Mode Decomposition prediction algorithm for a variety of model parameters.
The change in information is used to exhibit differential behavior between good model parameters and bad.
